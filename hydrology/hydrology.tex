\documentclass{article}
\usepackage{hyperref}
\usepackage{graphicx}
\usepackage{geometry}
\usepackage{amsmath}
\newcommand\norm[1]{\left\lVert#1\right\rVert}
\usepackage{tikz}
\usepackage{float}
\tikzstyle{process} = [rectangle, minimum width=3cm, minimum height=1cm, text center    ed, draw=black, fill=orange!30]
\usetikzlibrary{shapes.geometric, arrows}
\tikzstyle{arrow} = [thick,->,>=stealth]
\geometry{
 a4paper,
 total={170mm,257mm},
 left=30mm,
 right=30mm,
 top=30mm,
 bottom=15mm,
 }
\begin{document}
\title{Hydrology}
\author{Zhi Li}
\maketitle

\section{23 Unsolved problems in Hydrology}
\subsection{Catergorize Problems}
    \begin{enumerate}
        \item Time variability and change
        \begin{itemize}
            \item Climate Change $\rightarrow$ Q1,2,3.
            \item Land-cover Changes on Hydrological Fluxes $\rightarrow$ Q4.
        \end{itemize}
        \item Space variability and scaling
        \begin{itemize}
            \item understanding the nature of spatial variability of hydrological fluxes $\rightarrow$ Q5.
            \item the relation betwrrn point-scale to catchment-scale $\rightarrow$ Q6.
        \end{itemize}
        \item Variability of extremes
        \begin{itemize}
            \item Detection, Attribute, and Characteristics of flood-rich and drought-rich periods $\rightarrow$ Q9.
            \item land-cover changes on floods and droughts $\rightarrow$ Q10.
            \item temporal variability theme and flow path $\rightarrow$ Q7,8.
            \item geomorphological process e.g. melting, landslides links with floods/droughts $\rightarrow$ Q11.
        \end{itemize}
        \item Interfaces in hydrology
        \begin{itemize}
            \item fluxes and flow paths across compartments including physical-chemical-biological interactions $\rightarrow$ Q12, 13.
            \item locally inter-compartment fluxes and address issues at regional scales with hyper-resolution, global hydrological modelling and data-driven methods e.g. groundwater recharge to oceans $\rightarrow$ Q13.
            \item interaction under spatial-temporal variations between compartments to contribute to the degradation of water quality in catchment scale $\rightarrow$ Q14.
            \item Water and health in a hydrological perspective $\rightarrow$ Q15.
        \end{itemize}
        \item Measurement and data
        \begin{itemize}
            \item digital solutions to hydrology e.g. camera to particle detection $\rightarrow$ Q16
            \item use of proxies, replacing few accurate data by less accurate data through data-mining $\rightarrow$ Q17.
            \item fusion of quantitative with non-quantitative data e.g. social-economy, landuse from crowd-sourcing data $\rightarrow$ Q18.
        \end{itemize}
        \item Modelling methods
        \begin{itemize}
            \item hydrological models adapted to changing conditions which is more process-based rather than calibration-based $\rightarrow$ Q19.
            \item model structure uncertainty $\rightarrow$ Q20.
        \end{itemize}
        \item Interfaces with society
        \begin{itemize}
            \item hydrological contribution to societal problems with water-societal interactions $\rightarrow$ Q21.
            \item water-environment-energy-food-heath nexus $\rightarrow$ Q22.
            \item human-water interactions of ancient civilisations from hydrology to earth system sciences $\rightarrow$ Q23.
        \end{itemize}
    \end{enumerate}
\end{document}
\end{article}
